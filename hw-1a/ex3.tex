\begin{exercise}
(Función zeta de Riemann)
\begin{enumerate}[label=\alph*)]
    \item Muestre que la función zeta de Riemann
    $$\zeta(z) = \sum_{n=1}^\infty n^{-z}$$
    converge absoluta y uniformemente en $\Re(z) \ge 1 + \varepsilon$ para todo $\varepsilon > 0$.
    
    (\textit{Sugerencia:} Use la prueba M de Weierstrass.)
    
    \item Muestre que $\zeta(z)$ es analítica para todo $z \in \C$ tal que $\Re(z) > 1$.
    
    (\textit{Sugerencia:} Use algún corolario del teorema de Morera.)
    
    \item Demuestre la siguiente identidad:
    $$\zeta(z) \Gamma(z) = \int_0^\infty \frac {x^{z-1} e^{-x}} {1 - e^{-x}} \, dx$$
    
    (\textit{Sugerencia:} Parta de la definición de $\Gamma$ como la integral
    $$\Gamma(z) = \int_0^\infty e^{-t} t^{z-1} \, dt$$
    y aplique el cambio de variable $t = nx$, con $n = 1, 2, 3, \dots$)
    
    \begin{enumerate}[label=\arabic*)]
        \item A partir de esta expresión, obtenga una continuación meromorfa de $\zeta(z)$ en $\C$.
        
        (\textit{Sugerencia:} Por un lado, considere la integral
        $$\int_1^\infty \frac {t^{z-1}} {e^t - 1} \, dt$$
        
        Por otro lado, expanda la serie de Laurent
        $$\frac 1 {e^t - 1} = \sum_{n=-1}^\infty c_n t^n$$
        cuya región de convergencia es $0 < |t| < 2\pi$.)
        
        \item Describa los polos de esta extensión.
    \end{enumerate}
\end{enumerate}
\end{exercise}

\begin{solution}
\leavevmode
\begin{enumerate}[label=\alph*)]
    \item Pongamos $p = 1 + \varepsilon$. Cada término de la $p$-serie convergente
    $$\sum_{n=1}^\infty n^{-p}$$
    es una cota uniforme para el término correspondiente de la serie
    $$\sum_{n=1}^\infty \left| n^{-z} \right| = \sum_{n=1}^\infty n^{-\Re(z)}$$
    en la región $\Re(z) \ge p$. Por ende, la función zeta de Riemann
    $$\zeta(z) = \sum_{n=1}^\infty n^{-z}$$
    converge absoluta y uniformemente en la región $\Re(z) \ge p$.
    
    \item Sea $C$ una curva cerrada en la región $\Re(z) > 1$. Puesto que $C$ es compacta, la coordenada real $\Re(z)$ alcanza un valor mínimo $p = 1 + \varepsilon$ entre los puntos de $C$. Entonces $C$ está contenida en la región $\Re(z) \ge p$. Llamemos $L$ a la longitud de $C$ y observemos que la integral
    $$
    \int_C \sum_{n=1}^\infty \left| n^{-z} \right| \, dz
        \le \int_C \sum_{n=1}^\infty n^{-p} \, dz
        = L \cdot \sum_{n=1}^\infty n^{-p}
    $$
    es una cantidad finita. Entonces, por el teorema de Fubini-Tonelli,
    $$
    \int_C \zeta(z) \, dz
        = \int_C \sum_{n=1}^\infty n^{-z} \, dz
        = \sum_{n=1}^\infty \int_C n^{-z} \, dz
        = \sum_{n=1}^\infty 0
        = 0
    $$
    
    Entonces, por el teorema de Morera, $\zeta$ es una función holomorfa.
    
    \item Expandiendo las definiciones de $\zeta$, $\Gamma$ y sustituyendo $t = ns$, tenemos
    $$
    \zeta(z) \Gamma(z)
        = \sum_{n=1}^\infty n^{-z} \int_0^\infty e^{-t} t^{z-1} \, dt
        = \sum_{n=1}^\infty \int_0^\infty e^{-ns} s^{z-1} \, ds
    $$
    
    Pongamos $x = \Re(z)$ y observemos que la integral
    $$
    \zeta(x) \Gamma(x)
        = \sum_{n=1}^\infty \int_0^\infty e^{-nx} s^{x-1} \, ds
        = \sum_{n=1}^\infty \int_0^\infty \left| e^{-nz} s^{z-1} \right| \, ds
    $$
    es una cantidad finita. Entonces, por el teorema de Fubini-Tonelli,
    $$
    \zeta(z) \Gamma(z)
        = \int_0^\infty s^{z-1} \sum_{n=1}^\infty e^{-ns} \, ds
        = \int_0^\infty s^{z-1} \frac {e^{-s}} {1 - e^{-s}} \, ds
        = \int_0^\infty \frac {t^{z-1}} {e^t - 1} \, dt
    $$
    que es el resultado solicitado.
    
    \begin{enumerate}[label=\arabic*)]
        \item Reescribamos el resultado anterior como
        $$
        \zeta(z) \Gamma(z)
            = \int_0^1 \frac {t^{z-1}} {e^t - 1} \, dt
            + \int_1^\infty \frac {t^{z-1}} {e^t - 1} \, dt
        $$
        
        Dado un compacto $K \subset \C$, existe un instante uniforme $T \ge 1$ a partir del cual el integrando es dominado en módulo por $e^{-t/2}$ para todo $z \in K$. Entonces la integral
        $$\int_T^\infty \left| \frac {t^{z-1}} {e^t - 1} \right| \, dt \le \int_T^\infty e^{-t/2} \, dt$$
        es una cantidad finita. Por ende, la integral en el tramo final
        $$F(z) = \int_1^\infty \frac {t^{z-1}} {e^t - 1} \, dt$$
        converge de manera localmente uniforme a una función entera de $z$.
        
        Consideremos la serie de Laurent centrada en el polo simple $t = 0$ del factor meromorfo
        $$f(t) = \frac 1 {g(t)} = \frac 1 {e^t - 1} = \sum_{n=-1}^\infty c_n t^n$$
        
        Esta serie converge de manera localmente uniforme en $0 < |t| < 2\pi$, pues los polos más cercanos a $t = 0$ son $t = \pm 2\pi i$. Por ende, la integral en el tramo inicial es
        $$G(z) = \int_0^1 \frac {t^{z-1}} {e^t - 1} \, dt = \int_0^1 \sum_{n=-1}^\infty c_n t^{z+n-1} \, dt$$
        
        Pongamos $x = \Re(z)$ y observemos que, en la región $x > 1$, la integral
        $$
        G(x)
            = \int_0^1 \sum_{n=-1}^\infty c_n t^{x+n-1} \, dt
            = \int_0^1 \sum_{n=-1}^\infty c_n \left| t^{z+n-1} \right| \, dt
        $$
        es una cantidad finita. Entonces, por el teorema de Fubini-Tonelli,
        $$
        G(z)
            = \int_0^1 \sum_{n=-1}^\infty c_n t^{z+n-1} \, dt
            = \sum_{n=-1}^\infty\int_0^1 c_n t^{z+n-1} \, dt
            = \sum_{n=-1}^\infty \frac {c_n} {z+n}
        $$
        
        Recordemos que la serie $f(1) = c_{-1} + c_0 + c_1 + \dots$ es absolutamente convergente. Tomemos un entero $k \ge 2$ y un radio ligeramente menor $r = k - \varepsilon$. Cada término de la serie
        $$\frac 1 \varepsilon \cdot \sum_{n=k}^\infty |c_n|$$
        es una cota uniforme para el término correspondiente de la serie
        $$\sum_{n=k}^\infty \left| \frac {c_n} {z+n} \right|$$
        en la región $|z| < r$. Por ende, el sufijo
        $$G_k(z) = \sum_{n=k}^\infty \frac {c_n} {z+n}$$
        es una función holomorfa en $|z| < k$. Por ende, $G$ es una función meromorfa en $\C$ y tiene las mismas singularidades que el prefijo
        $$G(z) - G_k(z) = \frac {c_{-1}} {z-1} + \dots + \frac {c_{k-1}} {z+k-1}$$
        en cada disco $|z| < k$. Estas singularidades son polos simples en los puntos $z = -n$ con $c_n \ne 0$ y los residuos correspondientes son $\Res(G, -n) = c_n$.
        
        Finalmente, la función $\zeta$ puede ser extendida de manera meromorfa en $\C$ como
        $$\zeta = \frac {F + G} \Gamma$$
        
        \item Puesto que $F$, $1/\Gamma$ son funciones enteras, todo polo de $\zeta$ se manifiesta como polo de $G$. Puesto que $G$ sólo tiene polos simples, $\zeta$ y $\Gamma$ no tienen polos comunes. Entonces el único polo de $\zeta$ es $z = 1$ y el residuo correspondiente es
        $$
        \Res(\zeta, 1)
            = \frac {\Res(G, 1)} {\Gamma(1)}
            = c_{-1}
            = \Res(f, 0)
            = \frac 1 {g'(0)}
            = 1
        $$
    \end{enumerate}
\end{enumerate}
\end{solution}
