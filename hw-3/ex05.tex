\begin{exercise}
Sea $\Omega^\star(\R^n)$ el espacio de formas diferenciales en $\R^n$ y sea $d : \Omega^\star(\R^n) \to \Omega^\star(\R^n)$ el operador derivada exterior. Pruebe las siguientes afirmaciones:
\begin{enumerate}[label=\alph*)]
    \item $d(\alpha \wedge \beta) = d\alpha \wedge \beta + (-1)^p \, \alpha \wedge d\beta$, para todo $\alpha \in \Omega^p(\R^n)$, $\beta \in \Omega^q(\R^n)$.
    
    \item $d^2 = d \circ d = 0$, esto es, $\Omega^\star(\R^n)$ es un complejo de cadenas.
\end{enumerate}
\end{exercise}

\begin{solution}
\leavevmode
\begin{enumerate}[label=\alph*)]
    \item Por inducción, en el grado total de $\alpha \wedge \beta$, tenemos los siguientes casos:
    \begin{itemize}
        \item Si $\alpha = f$ y $\beta = g$ son funciones, entonces
        $$
        d(fg)
            = \p {} {x^i} (fg) \, dx^i
            = g \, \p f {x^i} \, dx^i + f \, \p g {x^i} dx^i
            = g \, df + f \, dg
        $$
        
        \item Si $\alpha = \alpha_i \wedge dx^i$ es de grado positivo, entonces
        $$
        d(\alpha \wedge \beta)
            = (-1)^{pq} \, d(\beta \wedge \alpha)
            = (-1)^{pq+q} \, \beta \wedge d\alpha + (-1)^{pq} \, d\beta \wedge \alpha
            = d\alpha \wedge \beta + (-1)^p \, \alpha \wedge d\beta
        $$
        
        \item Si $\beta = \beta_i \wedge dx^i$ es de grado positivo, entonces
        \begin{align*}
        d(\alpha \wedge \beta)
            &= d(\alpha \wedge \beta_i \wedge dx^i) \\
            &= d(\alpha \wedge \beta_i) \wedge dx^i \\
            &= [d\alpha \wedge \beta_i + (-1)^p \, \alpha \wedge d\beta_i] \wedge dx^i \\
            &= d\alpha \wedge \beta_i \wedge dx^i + (-1)^p \, \alpha \wedge d\beta \wedge dx^i \\
            &= d\alpha \wedge \beta + (-1)^p \, \alpha \wedge d\beta
        \end{align*}
    \end{itemize}
    
    \item Por inducción en el grado de $\omega \in \Omega^k(\R^n)$, tenemos los siguientes casos:
    \begin{itemize}
        \item Si $\omega = f$ es una función, entonces\footnote{La contracción de un tensor simétrico con uno antisimétrico es cero.}
        $$d^2 f = \m f {x^i} {x^j} \, dx^i \wedge dx^j = 0$$
        
        \item Si $\omega = \omega_i \wedge dx^i$ es de grado positivo, entonces
        $$d^2 \omega = d^2 (\omega_i \wedge dx^i) = d^2 \omega_i \wedge dx^i = 0$$
    \end{itemize}
\end{enumerate}
\end{solution}
