\begin{exercise}
Sea $X$ una superficie de Riemann compacta. Sean $p_0, \dots, p_n \in X$ puntos distintos y sean $w_0, \dots, w_n \in \C$ valores arbitrarios. Construya una función meromorfa $\varphi : X \to \widehat \C$ que envía $\varphi(p_i) = w_i$.
\end{exercise}

\begin{remark}
No es necesaria la hipótesis de que $w_0, \dots, w_n$ son distintos.
\end{remark}

\begin{solution}
Sea $m = 1 + 2g + n$, donde $g$ es el género de $X$. Por construcción, el divisor
$$D = mp_0 - (p_1 + \dots + p_n)$$
excede en grado al divisor canónico de $X$. Entonces, por el teorema de Riemann-Roch,
$$\dim_\C \L_D = 1 - g + \deg D =  2 + g$$
existen funciones meromorfas no constantes\footnote{Si hubiésemos tomado $m = 2g + n$, esta parte de la prueba no funcionaría en el caso $g = n = 0$.} con un único polo en $p_0$ y ceros en $p_1, \dots, p_n$. Compongamos cualquiera de ellas con una transformación de Möbius que fija $0$ y envía $\infty \mapsto w_0$. Entonces tenemos una función meromorfa $\varphi_0 : X \to \widehat \C$ tal que
$$
\varphi_0(p_i) =
    \begin{cases}
        w_0, & \text{si } i = 0 \\
        0, & \text{si } i \ne 0
    \end{cases}
$$
Análogamente, para cada $i = 1, \dots, n$, tenemos una función meromorfa $\varphi_i : X \to \widehat \C$ tal que
$$
\varphi_i(p_j) =
    \begin{cases}
        w_i, & \text{si } j = i \\
        0, & \text{si } j \ne i
    \end{cases}
$$
Entonces $\varphi = \varphi_0 + \dots + \varphi_n$ es la función solicitada.
\end{solution}
