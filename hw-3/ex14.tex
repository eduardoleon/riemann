\begin{exercise}
Sea $X$ una superficie de Riemann. Sean $f, g$ funciones diferenciables tales que (a) al menos una es armónica y (b) al menos una tiene soporte compacto. Pruebe que $\langle f, g \rangle_D = 0$.
\end{exercise}

\begin{remark}
No es necesaria la hipótesis de que $X$ es compacta.
\end{remark}

\begin{solution}
Si $f$ es la función armónica, pongamos $\omega = -g \, \partial f$. Entonces,
$$
d\omega
    = -\bar \partial (g \, \partial f)
    = \partial f \wedge \bar \partial g - \cancelto 0 {g \, \bar \partial \partial f}
$$
Si $g$ es la función armónica, pongamos $\omega = f \, \bar \partial g$. Entonces.
$$
d\omega
    = \partial (f \, \bar \partial g)
    = \partial f \wedge \bar \partial g + \cancelto 0 {f \, \partial \bar \partial g}
$$
Sea cual fuere el caso, el producto interno de Dirichlet se reduce a
$$\langle f, g \rangle_D = \int_X 2i \, \partial f \wedge \bar \partial g = \int_X 2i \, d\omega$$
Por construcción, $\omega$ tiene soporte compacto. Sea $U \subset X$ una vecindad del soporte tal que $U$ misma tiene clausura compacta. Entonces $\omega$ se anula en la frontera $\partial U$. Por el teorema de Stokes,
$$\langle f, g \rangle_D = \int_X 2i \, d\omega = \int_U 2i \, d\omega = \int_{\partial U} 2i \, \omega = 0$$
\end{solution}
