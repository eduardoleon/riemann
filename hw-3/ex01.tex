\begin{exercise}
Sea $X$ una superficie de Riemann. Pruebe que el fibrado tangente
$$TX = \bigcup_{p \in X} \{ p \} \times T_pX$$
es un fibrado vectorial real diferenciable de dimensión 2.
\end{exercise}

\begin{solution}
Tomemos un atlas diferenciable sobre $X$. Para cada vecindad $U \subset X$ con una carta coordenada $\varphi : U \to \R^2$, sea $TU$ la parte de $TX$ que se proyecta sobre $U$ y definamos $T\varphi : TU \to \R^4$ por
$$T\varphi(p, v_p) = (\varphi(p), (\varphi \circ \gamma)'(0))$$
donde $\gamma : (-\varepsilon, \varepsilon) \to S$ es una curva que pasa por $\gamma(0) = p$ con velocidad $v_p$.

Dada una segunda carta $\psi : U \to \R^2$, sin pérdida de generalidad definida sobre la misma vecindad $U$, consideremos la función de transición $\tau : \varphi(U) \to \psi(U)$. Por la regla de la cadena,
$$(\psi \circ \gamma)'(0) = (\tau \circ \varphi \circ \gamma)'(0) = J\tau(z) \cdot (\varphi \circ \gamma)'(0)$$
donde $z = \varphi(p)$. Entonces la función de transición $T\tau : \varphi(U) \times \R^2 \to \psi(U) \times \R^2$ en el fibrado es
$$T\tau(z, v) = (\tau(z), J\tau(z) \cdot v)$$

Por construcción, las funciones de transición en el fibrado tangente son diferenciables, $\R$-lineales en las componentes vectoriales y conmutan con la proyección a la base. Entonces, el atlas sobre $TX$ define una estructura de fibrado vectorial real diferenciable de dimensión $2$ en cada fibra.
\end{solution}
